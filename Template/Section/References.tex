\nocite{*}
\renewcommand{\refname}{Bibliografia}
\begin{thebibliography}{00}
	\bibitem{accenti2015}
	E. Accenti, \emph{Dalle Piramidi ai Microchip. Il Computer nella storia. Dal 4000 ac al 2000}, S.l., Edizioni Ettore Accenti, 2015.

	\bibitem{kempf1961}
	K. Kempf, \emph{Electronic Computers Within The Ordnance Corps}, FTP.ARL.ARMY.MIL, 1961. Available at: \href{http://ftp.arl.mil/~mike/comphist/61ordnance/chap2.html}{http://ftp.arl.mil/~mike/comphist/61ordnance/chap2.html}

	\bibitem{cringely1996}
	R. X. Cringely, \emph{Accidental Empires. How the Boys of Silicon Valley Make Their Millions, Battle Foreign Competition, and Still Can't Get a Date}, Second Edition, London, Penguin Books, 1996.

	\bibitem{mccartney1999}
	S. McCartney, \emph{ENIAC\@. The Triumphs and Tragedies of the World's First Computer}, New York, Walker Publish Company, Inc., 1999.

	\bibitem{stallings2006}
	W. Stallings, \emph{Computer Organization and Architecture Designing For Performance}, Eighth Edition, Upper Saddle River, Pearson Education, Inc., 2006 (trad. It.\ \emph{Architettura e Organizzazione dei Calcolatori. Progetto e Prestazioni}, Ottava Edizione, Pearson, 2010).

\end{thebibliography}