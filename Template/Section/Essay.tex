\section*{La Storia di Eniac}
\indent

Lo studio della Storia porta a concludere che ogni evento porta dei cambiamenti e in particolare quelli bellici portano sempre a delle innovazioni. La necessità di prevalere di uno Stato porta il suo Governo a investire ingenti risorse nella ricerca scientifica ed è in questo contesto che nel 1946, negli Stati Uniti d’America, viene sviluppato “ENIAC”, il primo computer elettronico di dimensioni enormi tanto da occupare uno spazio pari a una moderna palestra e contenente 18.000 tubi a vuoto più conosciuti come valvole termodinamiche. Il progetto di ENIAC ebbe origine dalla necessità di allegare alle nuove armi d’artiglieria una tabella di tiro. Questa era indispensabile per poter colpire un bersaglio posto a distanza, in quanto, riportava le istruzioni sull’angolazione di puntamento e di sparo per ogni singola arma tenendo in considerazione anche la tipologia di terreno in cui si svolgevano le azioni di guerra nei diversi Stati e Continenti. Il Ballistic Research Laboratory (BRL) ebbe per primo l’incarico di sviluppare queste tabelle di tiro e ogni singolo impiegato dell’azienda, munito di una moderna calcolante, lavorava quasi una giornata per crearne una sola. Nel 1942, per far fronte all’impegno di sviluppare le tabelle militari, il BRL strinse un accordo di cooperazione con la Moore School of Electrical Engineering of the University of Pennsylvania che disponeva di un analizzatore differenziale ossia di un computer analogico in grado di risolvere equazioni differenziali. L’Università per sostenere l’impegno assunto dovette assumere altri 200 dipendenti laureati in matematica, tutte donne, in quanto gli uomini erano impiegati in guerra, e malgrado questo molte armi non poterono essere inviate al fronte perché ancora prive della loro specifica tabella. Per risolvere il problema di sopperire alla mancanza di personale e velocizzare i tempi di fornitura delle tabelle di tiro, il docente di ingegneria elettrica John Mauchly e il suo allievo John Eckert proposero la costruzione di un calcolatore usando la tecnologia dei tubi a vuoto. Il 05 giugno 1943 l’Esercito accettò la proposta con la prospettiva di disporre di uno strumento di calcolo veloce e finanziò i primi sei mesi di lavoro con la consistente cifra 61.700 dollari. Era il progetto PX, chiamato da J.Mauchly e J. Eckert “Electronic Numerical Integrator” (Integratore Numerico Elettronico) e successivamente dal colonnello Paul N. Gillon “Electronic Numerical Integrator And Computer” (Calcolatrice e Integratore Numerico Elettronico) che poi coniò anche l’acronimo “ENIAC”. Lo stanziamento di un budget alto era finalizzato a mettere a punto un calcolatore moderno utile per lo sviluppo di tabelle di tiro per l’artiglieria ma necessario anche per completare il Progetto Y, ossia il Progetto Manhattan, che aveva come obiettivo la costruzione della bomba atomica. In quest’ultimo lavoro erano impegnate le più brillanti menti dell’epoca e i loro studi richiedevano un’elevata precisione nel calcolo. Il Progetto ENIAC fu guidato dal Capitano dell’esercito Herman Goldstine che non si limitò a supervisionare i Professori J. Mauchly e John G. Brainerd e J. Eckert ma contribuì con le sue conoscenze matematiche allo sviluppo del progetto stesso e inoltre seppe programmare adeguatamente il lavoro decidendone l’organizzazione e i tempi. All’inizio infatti la collaborazione tra gli studiosi era stata difficile essendo J. Eckert uno stacanovista che spesso discuteva con gli altri membri perché riteneva non si impegnassero abbastanza e J.Mauchly un solitario che si chiudeva nel suo studio e tornava solo dopo aver risolto il problema che si era posto e le pressioni esercitate dai committenti per un rapido sviluppo del progetto non facilitavano i loro rapporti.  Il team di progettazione poteva contare anche sulla consulenza degli scienziati impegnati a Los Alamos nel progetto Manhattan e di fatto sui migliori fisici e matematici dell’epoca tra cui John Von Neumann. Quest’ultimo fu l’ideatore del “stored-program concept” ossia del concetto di programma memorizzato che determinò l’innovativa struttura di ENIAC.  Una volta completato il nuovo computer, ENIAC, si rendeva necessaria la sua programmazione.  A tale scopo l’Università della Pennsylvania selezionò tra il suo personale il team che si sarebbe dovuto occupare di questo e, su oltre duecento candidati laureati in matematica, vennero scelte sei donne conosciute poi con il nome di “ENIAC Girls” che, di fatto, furono le prime programmatrici professioniste della storia: Kay McNulty, Betty Jennings, Betty Snyder, Marlyn Wescoff, Fran Bilas e Ruth Lichterman.  Il computer fu costruito, come proposto da J.V. Neumann, rispecchiando lo schema delle tre unità principali distinte ognuna programmata con una funzione specifica. Gli attuali calcolatori hanno ENIAC come capostipite, salvo rare eccezioni, e sono detti “macchine di Von Neumann” in quanto viene mantenuta la stessa struttura ideata dallo scienziato. La prima parte del computer è la “Centrale Aritmetica” ossia un calcolatore in grado di eseguire le operazioni aritmetiche elementari ossia l’addizione, la sottrazione, la moltiplicazione e la divisione e ma anche di supportare operazioni complesse come la radice quadrata. La seconda parte è la “Memoria” ossia un’area dedicata esclusivamente allo stoccaggio di dati e del codice necessario per eseguire determinate istruzioni. La terza parte chiamata “Controllo Centrale” prevede il controllo logico del dispositivo ossia la corretta sequenza con cui eseguire le operazioni. Bisogna tuttavia distinguere tra le istruzioni relative a un specifico problema e quelle per il controllo del flusso, che hanno lo scopo di garantire il corretto funzionamento. Per realizzare ciò è necessaria avere una piccola memoria per lo stoccaggio delle prime, mentre le seconde vengono eseguite su più dispositivi dedicati, ognuno con un compito preciso. Per la realizzazione del computer il gruppo lavorò di giorno e di notte, sfruttando la luce del sole per costruire la macchina e il buio della notte per discutere il suo funzionamento logico. Il primo componente a essere sviluppato fu un accumulatore in grado di lavorare su base decimale e quindi di contenere 20 numeri da 10 cifre e, per aumentare la velocità, di compiere le operazioni di addizione e di sottrazione senza delegare il compito a un altro componente. Tutti i dispositivi erano progettati per comunicare tra loro, per scambiarsi i dati e i risultati, generavano degli impulsi e ogni segnale veniva trasmesso lo stesso numero di volte della sua rappresentazione e inviato nel canale corrispondente alla sua unità decimale.  Per ottenere una maggior velocità nei calcoli i progettisti eliminarono tutte le parti meccaniche per la rappresentazione dei numeri e le sostituirono con i tubi a vuoto, attualmente conosciuti come valvole termodinamiche, che vengono attivati mediante impulsi elettrici e che indicano le varie cifre mediante il proprio stato di acceso e spento.  A progetto completato ENIAC contava 18.000 tubi a vuoto e dissipava in calore una potenza termica di circa 200 kW. Questo costituiva un grande problema perché il calore generato, soprattutto nella fase di accensione e di spegnimento, era tale da bruciare una valvola termodinamica con la frequenza di una ogni due minuti. Il problema del surriscaldamento del computer era talmente importante che a seguito di una rottura di una ventola di raffreddamento una parte dello stesso prese fuoco fortunatamente senza causare danni. Nel 1948, a progetto già completato, i tubi a vuoto vennero sostituiti con delle valvole termodinamiche di qualità superiore riducendo una loro rottura a: una valvola ogni due giorni. ENIAC tuttavia per funzionare richiedeva 174 kilowatt e costava 650 dollari di corrente all’ora anche solo per essere lasciato in stand by. La scelta di operare in base decimale e non in binario, come i computer moderni, fu condizionata dall’elevato numero di tubi a vuoto che sarebbero stati necessari per operare in base due e che pertanto avrebbero aumentato notevolmente i costi del computer stesso. La costruzione pratica di ENIAC non fu la parte più difficile o quella che richiese il maggior tempo. La vera sfida per il Prof. Mauchly e i suoi Collaboratori fu la velocità del sistema perché la rapidità nel fare i calcoli era l’obiettivo primario del committente. Per ottenere il risultato cercato dovettero pertanto riuscire a far cooperare al meglio tutti i dispositivi del computer perché era uno spreco avere a disposizione un componente in grado di risolvere calcoli in pochi secondi se poi per trasferire i risultati e salvarli in memoria veniva richiesto il triplo del tempo creando il classico collo di bottiglia che avrebbe rallentato tutto il sistema. Per evitare questo il team al lavoro dovette rivedere il progetto e le componenti che rallentavano la macchina. Nella maggior parte dei casi fu sufficiente migliorare la qualità delle componenti ma in alcuni casi fu necessario effettuare una nuova progettazione strutturale.  La modifica più efficace e consistente fu il passaggio dalle schede perforate a uno switch elettronico per rappresentare le costanti nei calcoli. L’eliminazione dell’input cartaceo accelerò di molto la velocità di ENIAC ma questo richiese dei tempi di programmazione più lunghi. Con le nuove modifiche, la preparazione di un programma necessitava in genere di uno o due mesi a cui si doveva aggiungere fino a un’altra settimana di lavoro per correggere tutti i possibili errori e altri uno o due giorni per la preparazione del computer a eseguire un programma poiché queste operazioni potevano coinvolgere fino a 3000 interruttori da regolare. Il 2 settembre 1945 la seconda guerra mondiale finì ed ENIAC, per la sua complessità di progettazione e programmazione, non era ancora terminato e richiese ancora qualche mese di lavoro per essere ultimato. L’hardware funzionava perfettamente e i programmi erano tutti impostati e, anzi, le sei programmatrici avevano acquisito perfino la competenza di riconoscere e di sostituire le componenti che si guastavano. Dopo 200.000 ore di lavoro e un costo di 486.804,22 dollari il Professor Mauchly consegnò all’esercito il computer anche se purtroppo essendo stato ENIAC completato a guerra ormai finita non poteva essere più utilizzato per la compilazione delle tabelle di tiro per d’artiglieria. L’Esercito, con lo scopo di aumentare la propria popolarità, decise di farlo conoscere e di mostrarlo al pubblico e il 15 febbraio 1946, con una solenne cerimonia ENIAC fu presentato insieme ai suoi due inventori Dr. John Mauchly e J. Presper Eckert ma furono escluse le sei programmatrici. ENIAC fu acclamato dalla stampa come “Il Cervello Meccanico dell’Esercito” ed ebbe un notevole successo tra i cittadini degli Stati Uniti d’America. L’appellativo “Cervello Meccanico” era più che corretto, infatti diversamente da tutte le altre macchine costruite fino a quel momento ENIAC era in grado non solo di ricevere dati ma anche di essere programmato e quindi di fatto gli si poteva insegnare cosa fare. Per questo motivo ENIAC fu subito utilizzato per risolvere i calcoli degli scienziati di Los Alamos che lavoravano alla realizzazione della Bomba H, conosciuta anche come bomba ad idrogeno, teorizzata da Edward Teller durante lo sviluppo della bomba atomica. Nell’inverno tra il 1946 e il 1947 ENIAC fu smantellato dalla Moore School per essere trasferito nella base militare di Aberdeen Proving Ground e a causa di questo trasferimento ritornò operativo solo nell’agosto di quell’anno. Durante tutta la guerra fredda, tra Stati Uniti d’America e Russia, eseguì diverse simulazioni di esplosioni nucleari, previsioni meteo per i lanci balistici e le tabelle di tiro per le nuove armi sviluppate. Continuò a lavorare fino al 1955 quando fu messo in pensione e fu definitivamente smontato. Alcune sue unità furono messe in esposizione nel Smithsonian Institution a Washington dove ancora possono essere viste. Le restanti furono lasciate ad arrugginirsi in un magazzino finché Arthur Burks, un consulente che contribuì alla costruzione di ENIAC, ne comprò due e, dopo averle restaurate, le donò alla Moore School of Electrical Engineering che le mise in esposizione nell’atrio. ENIAC ha conquistato il suo posto nella storia perché i suoi ideatori e poi tutti quelli che collaborarono alla sua realizzazione riuscirono a dimostrare che non era un solo progetto dimostrativo ma un macchinario in grado di durare nel tempo per la sua capacità di essere istruito aprendo pertanto la porta allo sviluppo dei moderni computer.